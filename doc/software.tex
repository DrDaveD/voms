\documentclass[a4paper]{book}
\usepackage{color}
\usepackage{longtable}
\begin{document}
\begin{titlepage}
\title{The VOMS Software Suite:\\ An Installation and User's Guide}
\author{Vincenzo Ciaschini}
\end{titlepage}
\maketitle
\tableofcontents
\newpage
\chapter{Generalities}
\section{Getting the software}
Voms can be downloaded from the authoritative infnforge CVS at
http://infnforge.cnaf.infn.it, or the EGEE copy at
http://jra1mw.cvs.cern.ch:8180/cgi-bin/jra1mw.cgi/org.glite.security.voms
You can get the nightly rpms at
http://glite.web.cern.ch/glite/packages/.  You may skip the next
chapter if you downloaded the RPM version.

\section{Compiling the source}
After having downloaded and installed the source, go to the
\texttt{voms/} subdirectory and run \texttt{./configure}.
Apart from the usual standard options, there are four extra ones you
may be interested into:
\begin{description}
\item[] \textbf{--enable-docs=[yes/no]}  --- This option, enabled by
  default, specifies whether the documentation should or should not be
  generated.
\item[] \textbf{--with-debug}  --- This option, disabled by default,
  specifies that the source should be compiled with debug options,
  e.g. without optimizations and with the symbol table included.  It
  is not advised to use a version compiled with this switch for
  production.
\item[] \textbf{--with-globus-flavor=$<$flavor$>$} --- This option
  compiles the code against the specified flavor of Globus.  Please
  note that this means that the specified flavor should be installed
  on the compiling machine. The default is \verb|gcc32dbg|.
\item[] \textbf{--with-globus=$<$dir$>$} --- This option specifies the
  path under which the Globus toolkit has been installed.  The default
  value is \verb|/opt/globus|.
\end{description}

After that, a simple \verb|make| is more than enough to compile the
sources.

\section{Installation}
To install the software you may execute the commands
%\verb|make install-client| to install the clients,
%\verb|make install-server| to install the servers, and
%\verb|make install-api| to install the API.  Alternatively, a simple
\verb|make install| will install all the components of the software.

\section{Compatibility}
With version 1.6.0 and onwards, compatibility with VOMS version 1.1.x
and previous version is now dropped.  This means that servers that are
not capable of generating ACs are now unsupported.

\chapter{voms}
\section{Configuration}
To complete configuration of \verb|voms|, you are supposed to
execute the \newline\verb|voms_install_db| command.  It takes
the following options:

\begin{longtable}{lp{3in}}
\textbf{--mysql-home} & This option lets you specify the home
directory of mysql.  This information is usually included in the
\$MYSQL\_HOME environment variable, and if that is the case on your
machine then you do not need to specify this option.\\
\textbf{--db} & This is the name of the database that will contain
the information about the VO.  Its default name is ``voms'', and you
need to specify this information if and only if you are installing
multiple servers on the same machine.  Otherwise the default is
perfectly fine.\\
\textbf{--port} & This is the port number where the VOMS server will
be listening.  There is no default value for this option, although
15000 is the recommended choice for the first server installed on a
host, and additional servers may use 15001, 15002, etc\ldots\\
\textbf{--voms-vo} & This is the name of the VO to which the VOMS
server belongs.  Its default value is ``unspecified'' which is
\emph{not} a valid value for a VO name.  For this reason, this option
should be \emph{always} specified.\\
\textbf{--mysql-admin} & This is the name of the MySQL root user.  It is
needed because the script needs to create a new DB and a new user. Its
default value is ``root'', which is the standard on the default MySQL
installation.\\
\textbf{--mysql-pwd} & This is the password of the MySQL account
specified by the previous option.  Its default value is ``'', meaning
that there is no password.  This is \emph{not} advisable.  The root
account of a MySQL server hosting a VOMS DB \emph{must} be protected
by a password.\\
\textbf{--voms-name} & This is the username of the voms MySQL account
that will be setup to access the newly created DB.  Its default value
is ``voms'', and it is perfectly fine if you are installing a single
server.  If you are installing further servers on the same machine, you
\emph{MUST} change this name to some other value.\\
\textbf{--voms-pwd} & This is the password associated with the
\emph{voms-name} account.  It does have a default value, but this
should never be used.  You should always specify a new value.\\
\textbf{--code} & This is a unique code for each server installed on the
same host. It is a value between 0 and 65535, and its default is the
value of \textbf{--port}.\\
\textbf{--db-type} & This specifies the type of db that will be used by
the server.  Currently accepted values are \emph{mysql} and \emph{oracle}.
There is no default for this option.\\
\textbf{--sqlloc} & This specifies the full path to the DB interface library.
Again, there is no default for this option.\\
\textbf{--compat} & This option must be specified if you plan to use
voms 1.5.x on a MySQL backend with an old version of voms-admin.  It
requires --db-type to be \emph{mysql}.\\
\textbf{--newformat} & This forces the server to generate ACs in the new 
(correct) format.  This is meant as a compatibility feature to ease migration 
while the servers upgrade to the new version.\\
\end{longtable}

A couple example invocations follows:

for the first VO.

voms-install-db --port 15000 --vo-name my-vo --mysql-pwd 'some' --voms-pwd 'thing'

for a second VO on the same host.

voms-install-db --db new-vo --port 15001 --vo-name new-vo --mysql-pwd
'some' --voms-name 'voms2' --voms-pwd 'thing' --code 1

The server also needs to have an host certificate installed.  Obtain
it from your CA using the CA-specific procedures, and then copy the
certificate in \verb|/etc/grid-security/hostcert.pem| and the private
key to \verb|/etx/grid-security/hostkey.pem|.  The owners should be
set to root.root for both files, and permission should be,
respectively, 644 and 600 or, better, 444 and 400.

\section{Server options}
Installing the server using the above described procedure should
correctly create a set of configuration files that will execute it
with the proper options.  However, there are many other options that
are not used by the default configuration script.  The following lines
will so describe the totality of the options.

\begin{longtable}{lp{2.9in}}
\textbf{--port}            & The port number on which the server
                             should be listening.  The default value
                             is 50000\\
\textbf{--vo}              & The name of the VO to which this server
                             will belong.  The default value is
                             ``unspecified''.\\ 
\textbf{--logfile}         & The location of the log file.  The
                             default location is
                             ``\$PREFIX/var/log/voms.$<$voname$>$''\\
\textbf{--globusid}        & The value of the GLOBUSID environment
                             variable.  There is no default value.\\ 
\textbf{--globuspwd}       & The value of the GLOBUSPWD environment
                             variable.  There is no default value\\ 
\textbf{--x509\_cert\_dir}   & The location where the CA certificates
                             are kept.  The default value is
                             /etc/grid-security/certificates\\ 
\textbf{--x509\_cert\_file}  & A file containing all the CA
                             certificates.  There is no default
                             value.\\ 
\textbf{--x509\_user\_proxy} & The location of the server's proxy.
                             There is no default value.\\
\textbf{--x509\_user\_cert}  & The location of the server's
                             certificate.  The default value is
                             ``/etc/grid-security/hostcert.pem''\\ 
\textbf{--x509\_user\_key}   & The location of the server's private
                             key.  The default value is
                             ``/etc/grid-security/hostkey.pem''\\ 
\textbf{--desired\_name}    & OBSOLETE.  This option will be removed in
                             the future.  Do \emph{not} use it.\\ 
\textbf{--foreground}      & OBSOLETE.  This option will be removed in
                             the future.  Do \emph{not} use it.\\ 
\textbf{--username}        & The name of the user with which VOMS will
                             access the DB.  The default value is
                             ``voms''\\ 
\textbf{--dbname}          & The name of the DB that VOMS will use.
                             The default value is ``voms''.\\
\textbf{--timeout}         & The maximum length of validity of the ACs
                             that VOMS will grant. (in seconds)  The
                             default value is 24 hours\\ 
\textbf{--passfile}        & The location of the file containing the
                             password needed to access the DB. This
                             file should be owned by root and have
                             permissions set to 400.  There is no
                             default value.  If this option is not
                             specified, than the password will be
                             asked to the user during server startup.\\ 
\textbf{--uri}             & The URI that the server will publish for
                             himself.  The default value is
                             $<$hostname$>$:$<$port$>$.\\ 
\textbf{--globus}          & The version of Globus installed on the
                             server's host. Use 20 for Globus 2.0 or
                             Globus 2.1, and 22 for Globus 2.2 and
                             Globus 2.4.  The default value is 22.\\ 
\textbf{--version}         & Prints the version number and compilation
                             date and then exits.\\ 
\textbf{--backlog}         & Sets the backlog on the socket.  The
                             default value is 50.\\
%\end{longtable}
%\begin{longtable}{lp{3in}}
\textbf{--conf}            & Lets you specify a file from which
                             options will be loaded.  This file should
                             have exactly one option per line, and
                             option that do have values should be
                             specified in the format
                             ``option=value''.\\
\textbf{--code}            & This is a unique numeric code, between 0
                             and 65535, used to identify different
                             servers installed on the same machine.
                             Its default value is the value of \textbf{--port}.\\
\textbf{--logtype}         & Chooses the type of messages that will be 
                             logged.  Possible values for this option are:
			                       \begin{description}
			                       \item[1] \emph{STARTUP} --- Messages during
			                         the startup phase.
			                       \item[2] \emph{REQUEST} --- Messages during
			                         the request processing phase.
			                       \item[4] \emph{RESULT} --- Messages during
			                         the result processing and sending phase.
			                       \end{description}
			                       The different possible values may be ORed together.
			                       The default value is 255.\\
\textbf{--loglevel}        & Sets the level of verbosity on log messages.
                             Its possible values are:
			                       \begin{description}
			                       \item[1] \emph{LEV\_NONE} --- Does not log
			                         anything.
			                       \item[2] \emph{LEV\_ERROR} --- Only log
			                         error messages.
			                       \item[3] \emph{LEV\_WARN} --- Also logs
			                         warning messages.
			                       \item[4] \emph{LEV\_INFO} --- Also logs
			                         informational messages.
			                       \item[5] \emph{LEV\_DEBUG} --- Also logs
			                         debug messages. This also sets the
				                       \emph{-logtype} options to 255.
			                       \end{description}
			                       Higher levels of verbosity include all messages
			                       from the lower levels. The default value for this 
			                       option is 2 (\emph{LEV\_ERROR}), also any value 
			                       higher than 5 is treated as 5 (\emph{LEV\_DEBUG})\\
\textbf{--logformat}       & This option sets the format for the log messages.
                             Its default value is ``\%d:\%h:\%s(\%p):\%V:\%T:\%F (\%f:\%l):\%m''.
			                       Details on the syntax will be given in the \emph{LOG Format}
			                       section below.\\
\textbf{--logdateformat}   & This option sets the format in which the date will be
                             printed.  It is the same format used by the
			                       \emph{strftime(3)} option, and its default value is ``\%c''.\\
\textbf{--debug}           & Slightly modifies the internal workings
                             of the server to ease debug. \emph{Never}
                             use it on production servers. Use of this
                             option is guaranteed to severely hurt
                             scalability.  This option also implies a
			                       \emph{--loglevel=5}.\\ 
\textbf{--sqlloc}          & This is the fully qualified path of the DB access library. 
                             Please note that there is no default to this option.\\
\textbf{--socktimeout}     & The maximum number of seconds that a server will wait 
                             on an inactive connection before dropping it.\\
\textbf{--maxlog}          & The maximum size of a single lock file.  Please note that
                             this size is approximate and may be exceeded by a few thousand
                             bytes.  Whenever this amount is exceeded, log files are rotated.
                             The default value is 10M.\\
\textbf{--newformat}       & This forces the server to generate ACs in the new 
                             (correct) format.  This is meant as a compatibility 
                             feature to ease migration while the servers upgrade to
                             the new version.\\
\end{longtable}

\section{LOG Format}

The format used for logging can be specified by the user via a format
string passed to the \emph{--logformat} option.  This string has a
format similar to that used by the printf-family function.

All characters are copied into the output string unchanged, except for
substitution sequences, which have the following format:
\%[$<$ength$>$]$<$char$>$, where $<$length$>$ is optional and, if
specified, express the maximum length of the text that will be
substituted. Characters in excess will be silently truncated.

$<$char$>$, on the other hand, selects the type of substitution desired,
according to the following table:

\begin{longtable}{lp{4in}}
\textbf{\%} & Substitutes a plain \% character.\\
\textbf{d}  & Substitutes the date. The date format is specified by
              the \emph{-logdateformat} option.\\
\textbf{f}  & Substitutes the name of the file that logged the message.\\
\textbf{F}  & Substitutes the name of the function that logged the message.\\
\textbf{h}  & Substitutes the hostname of the machine hosting the service.\\
\textbf{l}  & Substitutes the line number of the code that logged the message.\\
\textbf{m}  & Substitutes the message proper.\\
\textbf{t}  & Substitutes the number of the message type. (see \emph{-logtype})\\
\textbf{T}  & Substitutes the name of the message type. (see \emph{-logtype})\\
\textbf{v}  & Substitutes the number of the message level. (see \emph{-logtype})\\
\textbf{V}  & Substitutes the name of the message level. (see \emph{-logtype})\\
\end{longtable}

\chapter{voms-proxy-init}
\section{Introduction}
This command is used to contact the VOMS server and retrieve an AC
containing user attributes that will be included in the proxy
certificates.


\section{Configuration}
\subsection{The vomsdir directory}
Since the attribute certificates that come with voms proxies include
the signature of the issuing hosts, it becomes necessary for both the
command line utilities and the APIs to have a way to access the
issuing host's own certificate.  For this reason, a directory, must be
setup.

Its default name is ``/etc/grid-security/vomsdir'', but this can be
overridden by setting the X509\_VOMS\_DIR envirinment variable.

For each VO supported on the host, a subdirectory must be created with
the same name as the VO.  In this subdirectory two kinds of files may
be present.

\begin{enumerate}
\item \texttt{\*.lsc} files.  These files contain data regarding the
signing certificate, in the format:

$<$DN$>$\newline
$<$CA$>$\newline
\ .\newline
\ .\newline
\ .\newline
$<$DN$>$\newline
$<$CA$>$\newline

Where each couple DN,CA is the corresponding couple in the signing
certificate's cert chain. The name of this file MUST be $<$voms host
name$>$.lsc where $<$voms host name$>$ is the output of the
\texttt{hostname} command in the VOMS server.  Such files will be
tested against the certificate chain included in the VOMS ac.

\item Otherwise you may have files containing the whole certificate
chain of the host.  There is no special requirement on the name,
except that it must not end with '\texttt{.lsc}'
\end{enumerate}



Second, in ``\$PREFIX/etc/vomses'' You should put a copy of the
\emph{vomses} file distributed by all the VOMS servers you wish to
contact.  This subtree will be recursed into to examine all pertinent
files.

The easier way to comply to both previous points is to install the VO
config RPM that should be distributed by the VOMS servers themselves.

This is all the configuration that should be done for the use of this
command.

\section{Invocation}

The voms-proxy-init command can be invoked with the following
options: 

\begin{longtable}{lp{3in}}
\textbf{--voms}     & Specifies which server to contact.  The parameter
                      has the following syntax:
                      $<$alias$>$[:$<$command$>$] where $<$alias$>$ is
                      the alias of the server as specified in the
                      vomses files.  If the same alias is associated
                      to more than a single server, than those servers
                      are considered replicas of each other, and are
                      contacted in random order until one succeeds or
                      all fail. 

		      The [:$<$command$>$] part is optional.  If not
		      specified then the information returned will
		      include only group membership, while if you
		      specify :/Role=$<$rolename$>$ then you will also get the
		      role you asked for, provided that the server is
		      already prepared to grant it to you.  

		      Finally, if you specify
		      :/group/Role=$<$rolename$>$ as command, then you
		      will get the role \emph{rolename} in the group
		      \emph{/group} only, again granted that the
		      server is prepared to grant you that role.

		      This option can be specified multiple times, and
		      the operations will be carried out in the exact
		      order in which these options are specified in the
		      command line.\\
\textbf{--version}  & Prints version information and exits.\\
\textbf{--quiet}    & Prints only minimal informations.
		      \emph{WARNING}: some vital warnings may get
		      overlooked by this option.\\
\textbf{--verify}   & Verifies the certificate from which to create the
		      proxy.  This is not normally done, since in any
		      case, an invalid user certificate will be
		      detected when the proxy is actually used.\\
\textbf{--pwstdin}  & Specifies that the private key's passphrase
		      should be received from stdin instead than
		      directly from the console.\\
\textbf{--limited}  & Creates a limited certificate.\\
\textbf{--hours}    & Specifies the length of the validity of the
		      generated proxy, measure in hours.  The default
		      value is 12 hours.\\
\textbf{--bits}     & Specifies the length in bits of the private key
		      of the newly generated proxy certificate.  The
		      default value is 512.\\
\textbf{--cert}     & Specifies a non-standard location of the user's
		      certificate.  The default value is
		      ``\$X509\_USER\_CERT'' or, if this value is unset,
		      ``/\$HOME/.globus/usercert.pem''.\\
\textbf{--key}      & Specifies a non-standard location of the user's
		      private key.  The default value is
		      ``\$X509\_USER\_KEY'' or, if this value is unset,
		      ``\$HOME/.globus/userkey.pem''.\\
\textbf{--certdir}  & Specifies a non-standard location of the trusted
		      cert (CA) directory.  The default value is
		      ``/etc/grid-security/certificates''.\\ 
\textbf{--out}      & Specifies a non-standard location of the
		      generated proxy certificate.  The default value
		      is ``\$X509\_USER\_PROXY'' or, if this is empty,
		      ``/tmp/x509up\_u$<$id$>$'' where $<$id$>$ is the
		      user's UID.\\
\textbf{--order}    & This option specifies the order in which the
		      attributes granted by the VOMS servers should be
		      returned.  

		      The format of the parameter for this option is:
		      $<$group[:role]$>$, where ``group'' is a group name
		      and ``role'' is an (optional) role name.  This
		      option may be specified multiple times, to create
		      an ordered list of attributes.

		      Each server will receive this list, and will
		      strive to return the attributes he will grant in
		      the exact order specified by this list.  All
		      attributes not on this list will be returned in
		      an unspecified order, but after the recognized
		      attributes.  Also, should this list include an
		      attribute unknown to a specific server, such an
		      attribute will be simply ignored.

		      Finally, should a server be unable to grant the
		      first attribute of the list, it will return a
		      warning to the user.  However, this warning will
		      only be significant for the first server
		      contacted.\\
\textbf{--target}   & This option take advantage of the capability ACs
		      have to target themselves to a specific set of
		      receivers, so that only those receivers should,
		      in conforming implementation, act on the data
		      they get, while all others should reject it.

		      This options lets you specify a set of FQHNs,
		      each on a separate option, that will constitute
		      the set of targets for the generated AC.\\
\textbf{--vomslife} & This option lets you specify the validity, in
		      seconds, that you wish for the generated ACs.
		      Remember that this value has only an advisory
		      role.  VOMS servers may lower this duration if
		      the requested value exceeds the maximum they
		      have been configured to grant.  The default
		      value of this option is ``the value of the
		      --hours option.''\\
\textbf{--proxyver}  & The version of proxy certificate that will be generated.
                       May be 3 for new proxy certificate with critical 
                       Proxy Certificate Extension or 2 for old. When not specified
                       the version is decided upon underlying globus version.\\
\textbf{--policy}    & Specify the file containing the policy expression to put in
                       the PCI extension. The default is an empty policy expression.\\
\textbf{--policy-language} & Specify the language in which the policy is expressed. 
                             Two generic language are defined: id-ppl-inheritAll 
			     (default choice with an empty policy expression,
			     else invocated with IMPERSONATION\_PROXY or own OID), 
			     which indicates an unrestricted proxy that inherits all
			     rights from the issuing PI, and id-ppl-independent
                             (invocated with INDEPENDENT\_PROXY) which indicates an independent proxy that inherits no
			     rights from the issuing PI. \\
\textbf{--path-length}  & Specify the maximum depth of the path of proxy certificates
                          that can be signed by this proxy certificate.
			  A value of 0 means that this certificate must not be used to
			  sign a proxy certificate. If not present means that unlimited proxy can be signed.\\
\textbf{--globus}   & The version of Globus installed on the server's
		      host. Use 20 for Globus 2.0 or Globus 2.1, and
		      22 for Globus 2.2 and Globus 2.4.  The default
		      value is 22.\\  
\textbf{--noregen}  & For its normal workings, voms-proxy-init
		      first creates a proxy with which to contact the
		      VOMS servers, and then creates a new proxy to
		      hold all of the returned ACs.  This option skips
		      the creation of the first proxy, and assumes
		      that such a proxy already exists.\\
\textbf{--separate} & This option save the ACs in a separate file,
		      instead than including them into a proxy
		      certificate.\\ 
\textbf{--ignorewarn} & Specify this if you do not want to allow
		      warnings to be printed.\\
\textbf{--failonwarn} & Specify this if you want warnings to be
		      upgraded into errors.\\
\textbf{--confile}, 
\textbf{--userconf},
\textbf{--vomses} & These options specify the location of the vomses files or directories.
                    They should be either owned by the user, or by root. ``\$PREFIX/etc/vomses''
                    and ``\$HOME/.edg/vomses'' are added by default.  The three options are synonyms,
                    with one exception: --vomses may be specified any number of times..\\
\                 & COMPATIBILITY NOTE: This behaviour differs from the behaviour of previous versions,
                    where --confile was reserved for root-owned files, and --userconf was reserved for
                    user-owned files. This modification is backwards compatible and should solve all
                    the confusion problems. --userconf and --confile are now deprecated\\
\textbf{--conf}     & Lets you specify a file from which options will
		      be loaded.  This file should have exactly one
		      option per line, and option that do have values
		      should be specified in the format
		      ``option=value''.\\ 
\textbf{--debug}    & This option prints a series of additional debug
		      informations on stdout.  The additional output
		      returned by this option should \emph{always} be
		      included into bug reports for the
		      voms-proxy-init command.  User should not,
		      however, ever rely on informations printed by
		      this options.  Both content and format are
		      guaranteed to change between software
		      releases.\\
\textbf{--list} & Instead of producing an AC, this option prints on screen
          a list of all attributes available to the user.\\
\end{longtable}

\chapter{voms-proxy-info}
\section{Introduction}
This command is used to print to the screen the informations included
in an already generated VOMS proxy.

\section{Configuration}
The same as voms-proxy-init.

\section{Invocation}
\begin{longtable}{lp{3in}}
\textbf{--debug}    & This option prints a series of additional debug
		      informations on stdout.  The additional output
		      returned by this option should \emph{always} be
		      included into bug reports for the
		      voms-proxy-info command.  User should not,
		      however, ever rely on informations printed by
		      this options.  Both content and format are
		      guaranteed to change between software
		      releases.\\
\textbf{--version}  & Prints version information and exits.\\
\textbf{--conf}     & Lets you specify a file from which options will
		      be loaded.  This file should have exactly one
		      option per line, and option that do have values
		      should be specified in the format
		      ``option=value''.\\ 
\textbf{--file}     & This option lets you specify a non-standard
		      location of the user proxy.  The default value
		      is ``\$X509\_USER\_PROXY'' or, if this is empty,
		      ``/tmp/x509up\_u$<$id$>$'', where $<$id$>$ is
		      the user's UID.\\
\textbf{--subject}  & Prints the subject information.\\
\textbf{--issuer}   & Prints the issuer information.\\
\textbf{--type}     & Prints the proxy's type.\\
\textbf{--strength} & Prints the length (in bits) of the private
		      key.\\
\textbf{--valid}    & Prints the start and end validity times.\\
\textbf{--time}     & Prints the end validity as a number of seconds
		      for which the object will still be valid.\\
\textbf{--info}     & Lets ``--subject'', ``--issuer'', ``--valid''
		      and ``--time'' also apply to ACs, and prints
		      attributes values.\\
\textbf{--extra}    & Prints extra informations that were included in
		      the proxy.\\
\textbf{--all}      & Prints everything. (Implies all other
		      options.)\\
\textbf{--fqan}     & Specifies that attributes should be printed in
		      the FQAN format. (default)\\
\textbf{--extended} & Specifies that attributes should be printed in
		      the extended format.\\
\textbf{--exists}   & Activates the ``--hours'' and ``--bits''
		      options.\\
\textbf{--hours}    & Verifies that the proxy, and the ACs if
		      ``--info'' was specified, will be valid for at
		      least $<$H$>$ hours.\\
\textbf{--bits}     & Verifies that the proxy key has at least $<$B$>$
		      bits.\\
\end{longtable}


\chapter{voms-proxy-destroy}
\section{Introduction}
This command destroys an already existing VOMS proxy.

\section{Configuration}
No configuration needed.

\section{Invocation}
The following options may be used:

\begin{longtable}{lp{3in}}
\textbf{--debug}    & This option prints a series of additional debug
		      informations on stdout.  The additional output
		      returned by this option should \emph{always} be
		      included into bug reports for the
		      voms-proxy-info command.  User should not,
		      however, ever rely on informations printed by
		      this options.  Both content and format are
		      guaranteed to change between software
		      releases.\\
\textbf{--version}  & Prints version information and exits.\\
\textbf{--conf}     & Lets you specify a file from which options will
		      be loaded.  This file should have exactly one
		      option per line, and option that do have values
		      should be specified in the format
		      ``option=value''.\\ 
\textbf{--quiet}    & Prints only minimal informations.
		      \emph{WARNING}: some vital warnings may get
		      overlooked by this option.\\
\textbf{--file}     & This option lets you specify a non-standard
		      location of the user proxy.  The default value
		      is ``\$X509\_USER\_PROXY'' or, if this is empty,
		      ``/tmp/x509up\_u$<$id$>$'', where $<$id$>$ is
		      the user's UID.\\
\textbf{--dryrun}   & Only prints messages, but do not take any
		      actions.\\
\end{longtable}

\chapter{voms-proxy-fake}
\section{Introduction}
This command creates proxy certificates with fake ACs.  This is useful for test
purposes.

\section{Configuration}
No configuration is needed.

\section{Invocation}
\begin{longtable}{lp{3in}}

\textbf{--help}         & Displays usage.\\
\textbf{--version}      & Displays version.\\
\textbf{--debug}        & Enables extra debug output.  Note that the
                          exact format of this output is
                          version-dependent, and should not be relied
                          upon.\\
\textbf{-q}             & Quiet mode, minimal output.\\
\textbf{--verify}       & Verifies certificate used to make proxy.\\
\textbf{--pwstdin}      & Allows passphrase from stdin.\\
\textbf{--limited}      & Creates a limited proxy.\\
\textbf{--hours}        & The proxy is valid for the specified number
                          of hours.  The default values is 12 hours.\\
\textbf{--vomslife}     & Makes an AC with information valid for the
                          specified number of hours.  The default
                          value of 0 means ``as long as the proxy
                          certificate.''\\
\textbf{--bits}         & Number of bits in the key {512|1024|2048|4096}\\
\textbf{--cert}         & Non-standard location of the user certificate.\\
\textbf{--key}          & Non-standard location of the user key.\\
\textbf{--certdir}      & Non-standard location of the trusted
                          certificates directory.\\
\textbf{--out}          & Non-standard location of the new proxy
                          cert.\\
\textbf{--voms}         & Specifies the fake VOMS server that will
                          appear in the attribute certificate.  The
                          command part (the same as that of the
                          voms-proxy-init command) is ignored and is
                          present for compatibility with
                          voms-proxy-init.\\
\textbf{--include}      & Includes the specified file in the
                          certificate (in a non critical extension).\\
\textbf{--conf}         & Read options from the specified file.\\
\textbf{--policy}       & The file containing the policy expression.\\
\textbf{--policy-language} & The language in which the policy is
                             expressed.  Default is
                             IMPERSONATION\_PROXY.\\
\textbf{--path-length}  & Maximum depth of proxy certificate that can
                          be signed from this.\\
\textbf{--globus}       & Underlying Globus version.\\
\textbf{--proxyver}     & Version of the proxy certificate to create.
                          May be 2 or 3.  The default value is
                          dependent on the underlying globus version.\\
\textbf{--separate}     & Saves the voms credential on the specified
                          file.\\ 
\textbf{--hostcert}     & The cert that will be used to sign the AC.\\
\textbf{--hostkey}      & The key that will be used to sign the AC.\\
\textbf{--fqan}         & The string that will be included in the AC
                          as the granted FQAN.  No check is done on
                          the formal correctness of this string.\\
\textbf{--oldformat}    & This allows AC generation in the old (incorrect) format.\\
\end{longtable}

\chapter{voms-install-replica}
\section{Introduction}
This script allows a VOMS server to be setup as a slave of a master
host, so that it will automatically pickup all DB updates from the
master.  It only works for MySQL-based servers.

\section{Configuration}
Prior to using this script, the VOMS who has to become the master must
be configured.  The instructions to do so follow.

From the shell (just once):

\begin{verbatim}
cat >>/etc/my.cnf <<EOF
log-bin
server-id=1
EOF
\end{verbatim}

and then, every time a new replica must be created:

\begin{verbatim}
 mysql -p -e "grant super, reload , replication slave, replication
 client on voms_myvo.* to replica@'grid-se.pr.infn.it' identified by
 'replicapass'"

 mysql -p -e "grant select, lock tables on voms_myvo.* to
 replica@'grid-se.pr.infn.it'"
\end{verbatim}

The last two commands are meant to be specified on one line only, and
have been broken into multiple lines here for legibility.

You should obviously replace \texttt{replica@grid-se.pr.infn.it} and
\texttt{replicapass}  with the hostname of your new replica server and
the username that it will be using.  The username and the associated
password should then be communicated to the administrator of the
replica server. 

\section{Invocation}
After having configure the slave via the \emph{voms-install-db}
script, the \emph{voms-install-replica} script should be run.  It
takes the following options:

\begin{longtable}{lp{3in}}
\textbf{--db}                  & This is the name of the DB that will
                                 be replicated.  It must be the same
                                 name used in the master.\\
\textbf{--mysql-admin}         & This is the root account on the
                                 \emph{slave} DB.\\
\textbf{--mysql-pwd}           & This is the password associated to
                                 the \textbf{--mysql-admin} account.\\
\textbf{--master-host}         & This is the fully qualified hostname
                                 of the machine on which the server is
                                 running.\\ 
\textbf{--master-mysql-user}   & This is the username that the master
                                 administrator created for the replica
                                 to use.\\
\textbf{--master-mysql-pwd}    & This is the username that is
                                 associated to the
                                 \textbf{--master-mysql-pwd}
                                 account.\\
\textbf{--master-db}           & This is name of the DB that should be
                                 replicated.  It must be the same as
                                 the argument of the \textbf{--db}
                                 option.\\
\textbf{--master-log-file}     & This is the name of the copy of the
                                 master log file that will be created
                                 on the slave machine.\\
\textbf{--master-log-pos}      & This is the location in which the
                                 file specified by the
                                 \textbf{--master-log-file} will be
                                 placed.\\
\end{longtable}

\end{document}
