\documentclass[a4paper]{book}
\usepackage{color}
\usepackage{listings}
\newenvironment{compatibility}{\begin{quote}\color{red}Compatibility\begin{quote}}{\end{quote}\color{black}\end{quote}}
\newcommand{\also}[1]{\noindent \textbf{SEE ALSO}\newline\ \ \ \ \ #1}
\newcommand{\errors}{\noindent \textbf{ERRORS}\newline}
\newcommand{\results}{\noindent \textbf{RESULTS}\newline}
\newcommand{\result}{\noindent \textbf{RESULTS}\newline}
\newcommand{\return}{\noindent \textbf{RETURNS}\newline}
\newcommand{\parameter}[1]{\newline\textbf{#1}\ \ }
\begin{document}
\lstset{language=C}
\begin{titlepage}
\title{The VOMS C API\\ A Developer's Guide}
\author{Vincenzo Ciaschini}
\end{titlepage}
\maketitle
\tableofcontents
\newpage
\chapter{Introduction}
The VOMS API already come with their own documentation in doxygen
format.  However, that documentation is little more than a simple
enumeration of functions, with a very terse description.

The aim of this document is different.  Here the intention is not only
to describe the different functions that comprise the API, but also to
show how they are supposed to work together, what particular care the
user needs to take when calling them, what should be done to mantain
compatibility between the different versions, etc\ldots

Throughout this whole document, you will find sections marked thus:
\begin{compatibility}
Some information
\end{compatibility}
These section contain informations regarding both back and forward
compatibility between different versions of the API.

\begin{compatibility}
Finally, please note that everything not explicitly defined in this
argument should be considered a private detail and subject to change
without notice.
\end{compatibility}

\chapter{The API.}
There are three basic data structures: \verb|data, voms|
and \verb|vomsdata|.

\section{The data structure}
The first one, \verb|data| contains the data regarding a single
attribute, giving its specification in terms of Groups, Roles and
Capabilities. It is defined as follows:

{\begin{lstlisting}{}
struct data {
  char *group; 
  char *role;  
  char *cap;   
};
\end{lstlisting}}

All the values of these strings must be composed from regular
expression: \texttt{a-ZA-Z0-9\_/]*}.

\subsection{group}
This field contains the name of a group into which the user belongs.
The format of entries in this group is reminiscent of the structure of
pathnames, and is the following:
\begin{quote}
\begin{emph}
/group/group/.../group
\end{emph}
\end{quote}
where the name of the first group is by convention the name of the
Virtual Organization (VO), while each other \emph{/group} component is
a subgroup of the group immediately preceding it on the left. The
character '/' is not acceptable as part of a group name.

This field MUST always be filled.

\subsection{role}
This field contains the name of the role to which the user owns in the
group specified by \verb|group|.  If the user does not own any
particular role in that group, than this field contains the value
``NULL''.

\subsection{cap}
This field details a capability that the user has as a member of the
group specified by \verb|group| while owning the role specified by
\verb|role|.  If there is no specific capability, than this value is
``NULL''. 

No specific format is associated to a capability.  They are basically
free-form strings, whose value should be agreed between the AA and the
Attribute verifier.

\section{The voms structure}
The second one, \verb|voms| is used to group together all the
informations that can be gleaned from a single AC, and is defined as
follows:

{\begin{lstlisting}{}
#define  TYPE_NODATA 0  /*!< no data */
#define  TYPE_STD    1  /*!< group, role, capability triplet */
#define  TYPE_CUSTOM 2  /*!< result of an S command */


struct voms {
  int siglen;       
  char *signature;  
  char *user;       
  char *userca;     
  char *server;     
  char *serverca;   
  char *voname;     
  char *uri;        
  char *date1;      
  char *date2;      
  int   type;       
  struct data **std;
  char *custom;     
  int datalen;
  int version;
  char **fqan;
  char *serial;     
  /* Fields below this line are reserved. */
};
\end{lstlisting}}

The purpose of this structure is to present, in a readable format, the
data that has been included in a single Attribute Certificate
(AC).  While the various public fields may be freely modified to
simplify internal coding, such changes have no effect on the
underlying AC.  Let's examine the various fields in detail, starting
with the constructors.

\subsection{version}
This field specifies the version of this structure that is currently
being used.  A value of 0 indicates that it comes from an old format
extension, while a value of 1 indicates that this structure comes from
an AC.

\begin{compatibility}
Support for version 0 is going to be phased out of the code base in
roughly 6 months (late june - start of july).  When that happens,
version 0 structures will not be readable anymore. Until then, support
for it is being kept as a transition measure.
\end{compatibility}

Please do note that modifying the fields of a version 0 structure
associated with a \verb|versiondata| struct invalidates the result of
the \verb|VOMS_Export()| funciton on that object.

\subsection{siglen}
The length of the data signature.

\subsection{user}
This field contains the subject of the holder's certificate in
slash-separated format.

\subsection{userca}
This field contains the subject of the CA that issued the holder's
certificate, in slash-separated format.

\subsection{server}
This field contains the subject of the certificate that the AA used to
issue the AC, in slash-separated format.

\subsection{serverca}
This field contains, in slash-separated format, the subject of the CA that
issued the certificate that the AA used to issue the AC.

\subsection{voname}
This field contains the name of the Virtual Organization (VO) to which
the rest of the data contained in this structure applies.

\subsection{uri}
This is the URI at which the AA that issued this particular AC can be
contacted. Its format is:
\begin{quote}
\emph{fqdn}:\emph{port}
\end{quote}
where \emph{fqdn} is the Fully Qualified Domain Name of the server
which hosts the AA, while \emph{port} is the port at which the AA can
be contacted on that server.

\subsection{date1, date2}
These are the dates of start and end of validity of the rest of the
informations.  They are in a string representation readable to humans,
but they may be easily converted back to their original format, with a
little twist: dates coming from an AC are in GeneralizedTime format,
while dates coming from the old version data are in UtcTime format.

Here follows a code example doing that conversion:\bigskip\bigskip
{\begin{lstlisting}{}
ASN1_TIME *
convtime(char *data)
{
  char *data2 = strdup(data);

  if (data2) {
    ASN1_TIME *t= ASN1_TIME_new();

    t->data   = (unsigned char *)data2;
    t->length = strlen(data);
    switch(t->length) {
      case 10:
      t->type = V_ASN1_UTCTIME;
      break;
      case 15:
      t->type = V_ASN1_GENERALIZEDTIME;
      break;
      default:
      ASN1_TIME_free(t);
      return NULL;
    }
    return t;
  }
  return NULL;
}
\end{lstlisting}}

\subsection{type}
This datum specifies the type of data that follows.  It can assume the
following values:
\begin{description}
\item [TYPE\_NODATA] There actually was no data returned.
\begin{compatibility}
This is actually only true for version 0 structures. The following
versions will simply not generate a \verb|voms| structure in this
case.
\end{compatibility}
\item [TYPE\_CUSTOM] The data will contain the output of an ``S''
  command sent to the server.
\begin{compatibility}
Again, this type of datum will only be present in version 0
structures.  Due to lack of use, support for it has been disabled in
new versions of the server.
\end{compatibility}
\item [TYPE\_STD] The data will contain (group, role, capabilities)
  triples.
\end{description}

\subsection{std}
This vector contains all the attributes found in an AC, in the exact
same order in which they were found, in the format specified by the
\verb|data| structure.  It is only filled if the value of the
\verb|type| field is \verb|TYPE_STD|.
\begin{compatibility}
This structure is filled in both version 1 and version 0 structures,
although this is scheduled to be left empty after the transition
period has passed.
\end{compatibility}

\subsection{custom}
This field contains the data returned bu the ``S'' server command, and
it is only filled if the \verb|type| value id \verb|TYPE_CUSTOM|.

\subsection{fqan}
This field contains the same data as the \verb|std| field, but
specified in the Fully Qualified Attribute Name (FQAN) format.

\section{vomsdata}
The purpose of this object is to collect in a single place all
informations present in a VOMS extension.  All the fields should be
considered read-only.  Changing them has indefinite results.

{\begin{lstlisting}{}
struct vomsdata {
  char *cdir;
  char *vdir;
  struct voms **data;
  char *workvo;
  char *extra_data;
  int volen;
  int extralen;
  /* Fields below this line are reserved. */
};
\end{lstlisting}}

Let us see the fields in detail.

\subsection{data}
This field contains a vector of \verb|voms| structures, in the exact
same order as the corresponding ACs appeared in the proxy certificate,
and containing the informations present in that AC.

\subsection{workvo, volen}
\begin{compatibility}
This fields is obsolete in the current version.  Expect \verb|workvo|
to be set to \verb|NULL| and \verb|volen| to be set to 0.
\end{compatibility}

\subsection{extra\_data, extralen}
This field contains additional data that has been added by the user
via to the proxy via the \verb|-include| command option.  Extralen
represents the length of that data.

\subsection{cdir, vdir}
This fields contain the paths, respectively, of the CA certificates
and of the VOMS servers certificates.


\section{Functions}
\subsection{Generalities}
Most of these functions share two parameters,
\verb|struct vomsdata *vd|, and \verb|int *error|.  To avoid
repetition, these two parameters are described here.
\parameter{error}{This field contains the error code returned by one of the methods.
Please note that the value of this field is only significant if the
\emph{last} method called returns an error value.  Also, the value of
this field is subject to change without notice during method
executions, regardless of whether an error effectively occurred.

The possible values returned are:
VERR\_NONE, VERR\_NOSOCKET, VERR\_NOIDENT, VERR\_COMM, VERR\_PARAM, 
VERR\_NOEXT, VERR\_NOINIT, VERR\_TIME, VERR\_IDCHECK, VERR\_EXTRAINFO,
VERR\_FORMAT, VERR\_NODATA, VERR\_PARSE, VERR\_DIR, VERR\_SIGN, 
VERR\_SERVER, VERR\_MEM, VERR\_VERIFY, VERR\_TYPE, VERR\_ORDER, 
VERR\_SERVERCODE

In general, a first idea of what each code means can be gleaned from
the code name, but in any case every method description will document
which errors its execution may generate and on which conditions.}
\parameter{vd}{This parameter is a pointer to the vomsdata structure
  that should be used by the function for both configuration and data
  retrieval and also for data storage.}



\subsection{struct contactdata **VOMS\_FindByAlias(struct vomsdata *vd, char *alias, char *system, char *user, int *error)}

\begin{lstlisting}{}
struct contactdata { /*!< You must never allocate directly this structure. Its sizeof() is
		       subject to change without notice. The only supported way to obtain it
		       is via the VOMS_FindBy* functions. */
  char *nick;     /*!< The alias of the server */
  char *host;     /*!< The hostname of the server */
  char *contact;  /*!< The subject of the server's certificate */
  char *vo;       /*!< The VO served by this server */
  int   port;     /*!< The port on which the server is listening */                            
  char *reserved; /*!< HANDS OFF! */
  int   version;  /*!< The version of Globus on which this server runs. */
};
\end{lstlisting}

This function looks in the vomses files installed in both the
system-wide and user-specific directories for servers that have been
registered with a particular alias.
\parameter{alias}{The alias that will be searched for.  The search will be case
sensitive.}
\parameter{system}{The directory where the system-wide files are located.  
  If empty then its default is \verb|/opt/edg/etc/vomses|.}
\parameter{user}{The directory where the user-specific files are
  stored.  If empty its defaul is \verb|$VOMS_USERCONF|.  If this is
  also empty, then the default becomse \verb|$HOME/.edg/vomses|.}

\return
The return value is a NULL-terminated vector containing the data (in
\verb|contactdata| format) of all the servers known by the system
  that go by the specified alias.  This may be NULL if there was an
  error or no server was found registered with the specified alias.

The errors that you may find are:

\bigskip\begin{tabular}{lp{3in}}
VERR\_MEM  & Not enough memory.\\
VERR\_DIR  & There were some problems while traversing the directory.\\
VERR\_NONE & No error occurred. Simply, no servers were found.\\


\end{tabular}
\subsection{struct contactdata **VOMS\_FindByVO(struct vomsdata *vd,
  char *vo, char *system, char *user, int *error)}

This function looks in the vomses files installed in both the
system-wide and user-specific directories for servers that have been
registered as serving a particular alias.
\parameter{vo}{The alias that will be searched for.  The search will
be case sensitive.}
\parameter{system}{The directory where the system-wide files are
  located.  If this field is NULL then the default of
  \verb|/opt/edg/etc/cvomses| is used.}
\parameter{user}{The directory where the user-specific files are
  stored.  If this field is NULL, then the default of
  \verb|\$VOMS_USERCONF| is used.  If this is also empty, then the
  default becomse \verb|\$HOME/.edg/vomses|.}

\return
The return value is a NULL-terminated vector containing the data (in
\verb|contactdata| format) of all the servers known by the system
  that go by the specified VO.  This may be NULL if there was an
  error or no server was found registered with the specified VO.

The errors that you may find are:

\bigskip\begin{tabular}{lp{3in}}
VERR\_MEM  & Not enough memory.\\
VERR\_DIR  & There were some problems while traversing the directory.\\
VERR\_NONE & No error occurred. Simply, no servers were found.\\
\end{tabular}

\subsection{void VOMS\_DeleteContacts(struct contactdata **list)}

This function deletes a vector of server data returned by either the
\verb|VOMS_FindByAlias{}| or the \verb|VOMS_FindByVO()| functions.
This is the only supported way to deallocate the vector.  Any other
attempt will result in undefined behaviour.

It is although possible to deallocate only part of a vector.  See the
following code for an example.

\begin{lstlisting}{}
/*
 * Supposing that v is a vector returned by one of the VOMS_FindBy*()
 * functions.  Also suppose that n is the vector's size (including the
 * NULL ending element).
 *
 * The following snippet will delete just the first member.
 */
 struct contactdata *dummy[2];

 dummy[1] = NULL;
 dummy[0] = v[0];
 v[0]     = v[n-1];
 v[n-1]   = NULL;
 VOMS_DeleteContacts(dummy);
\end{lstlisting}

\ \parameter{list}{The data to be deleted.}

\return
None.

\subsection{struct vomsdata *VOMS\_Init(char *voms, char *cert)}

This function allocates and initializes a \verb|vomsdata| structure.
This is the only way to do so.  Trying to allocate a \verb|vomsdata|
structure by any other way will trigger undefined behaviour, since the
structure that is published is only a small part of the real one.
\parameter{voms}{The directory that contains the certificates of the
  VOMS servers.  If this value is NULL, then \verb|\$X509_VOMS_DIR| is
  considered.  If this is also empty than its default is
  \verb|/etc/grid-security/vomsdir|.}
\parameter{cert}{The directory that contains the certificates of the
  CAs recognized by the server.  If this value is NULL, then
  \verb|\$X509_CERT_DIR| is considered.  If this is also empty than its
  default is \verb|/etc/grid-security/certificates|.}


\return
A pointer to a properly initialized \verb|vomsdata| structure, or NULL
if something went wrong.  This is the only case in which an error code
would no be associated to the function.

The default values are strongly suggested.  If you want to hardcode
specific ones, think very hard about the less of configurability that
it would entail.

\subsection{struct voms *VOMS\_Copy(struct voms *, int *error)}

This function duplicates an existing \verb|voms| structure.  It is the
only way to do so.
\parameter{voms}{The voms structure that you wish to be duplicated.}

\result
A pointer to a voms structure that duplicates the content of the one
you passed, or NULL if something went wrong.

\errors
\bigskip\begin{tabular}{lp{3in}}
VERR\_MEM & Not enough memory.\\
\end{tabular}

\subsection{struct vomsdata *VOMS\_CopyAll(struct vomsdata *vd, int
  *error)}

This function duplicates an existing \verb|vomsdata| structure.  It is
the ONLY supported way to do so.

\result
A pointer to a voms structure that duplicates the content of the one
you passed, or NULL if something went wrong.

\errors
\bigskip\begin{tabular}{lp{3in}}
VERR\_MEM & Not enough memory.\\
\end{tabular}

\subsection{void VOMS\_Delete(strcut voms *v)}

This functions deletes an existing \verb|voms| structure.  It is the
ONLY supported way to do so.
\parameter{v}{A pointer to the \verb|voms| structure to delete. It is
  safe to call this structure with a NULL pointer.}

\result
None.

\subsection{int VOMS\_AddTarget(struct vomsdaa *vd, char *target, int
  *error)}

This function adds a target to the target list for the AC that will be
generated by a server when it will be contacted by the \verb|VOMS_Contact*()|
function. 
\parameter{target}{The target to add.  It should be a Fully Qualified
  Domain Name.}

\result
\begin{description}
\item[0] If something went wrong.
\item[$<>$0] Otherwise.
\end{description}

\errors
\bigskip\begin{tabular}{lp{3in}}
VERR\_NOINIT & The \verb|vomsdata| structure was not properly
initialized.\\
VERR\_PARAM  & The \verb|target| parameter was NULL.\\
VERR\_MEM    & There was not enough memory.\\
\end{tabular}

\subsection{void VOMS\_FreeTargets(struct vomsdata *vd , int *error)}

This function resets the list of targets. It always succeeds.  It is
also safe to call this function when targets have been set.

\subsection{char *VOMS\_ListTargets(struct vomsdata *vd, int *error)}

This function returns a comma separated string containing all the
targets that have been set by the \verb|VOMS_AddTarget()| function.
The caller is the owner of the returned string, and is responsible for
calling \verb|free()| over it when he no longer needs it.

\result
A string with the result, or NULL.

\bigskip\begin{tabular}{lp{3in}}
VERR\_NOINIT & The \verb|vomsdata| structure was not properly
initialized.\\
VERR\_MEM    & There was not enough memory.\\
\end{tabular}

\subsection{int VOMS\_SetVerificationType(int type, struct vomsdata
  *vd, int *error)}

This function sets the type of AC verification done by the
\verb|VOMS_Retrieve()| and \verb|Contact()| functions.  The choices
are detailed in the \verb|verify\_type| type.

\begin{lstlisting}{}
#define VERIFY_FULL      0xffffffff
#define VERIFY_NONE      0x00000000
#define VERIFY_DATE      0x00000001
#define VERIFY_NOTARGET  0x00000002
#define VERIFY_KEY       0x00000004
#define VERIFY_SIGN      0x00000008
#define VERIFY_ORDER     0x00000010
#define VERIFY_ID        0x00000020
\end{lstlisting}

The meaning of these types is the following:
\begin{description}
\item[VERIFY\_DATE] This flag verifies that the current date is
  within the limits specified by the AC itself.
\item[VERIFY\_TARGET] This flag verifies that the AC is being
  evaluated in a machine that is included in the target extension of
  the AC itself.
\item[VERIFY\_KEY] This flag is for a future extension and is unused at the
  moment. 
\item[VERIFY\_SIGN] This flag verifies that the signature of the AC
  is correct.
\item[VERIFY\_ORDER] This flag verifies that the attributes present
  in the AC are in the exact order that was requested.  Please note
  that this can ONLY be done when examining an AC right after
  generation with the Contact() function.  This flag is meaningless
  in all other cases.
\item[VERIFY\_ID] This flag verifies that the holder information
  present in the AC is consistent with:
\begin{enumerate}
\item The enveloping user proxy in case the AC was contained in one.
\item The user's own certificate in case the AC was received without
  an enclosing proxy.
\end{enumerate}
\item[VERIFY\_FULL] This flag implies all other verifications.
\item[VERIFY\_NONE] This flag disables all verifications.
\end{description}

These flags can be combined by OR-ing them together.  However, if
VERIFY\_NONE is OR-ed to any other flag, it can be dismissed, while if
VERIFY\_FULL is OR-ed to any other flag, all other flags ca be
dismissed.

If this function is not explicitly called by the user, a VERIFY\_FULL
flag is considered to be in effect.

\result
\begin{description}
\item[0] If there is an error.
\item[$<>$ 0] otherwise.
\end{description}

\bigskip\begin{tabular}{lp{3in}}
VERR\_NOINIT & The \verb|vomsdata| structure was not properly
initialized.\\
\end{tabular}

\subsection{int VOMS\_SetLifetime(int length, struct vomsdata *vd, int *error)}

This funxtion sets the requested lifetime for ACs that would be
generated as the result of a \verb|VOMS_Contact()| or
\verb|VOMS_ContactRaw()}|
  request.  Note however that this is only an hint sent to the server,
  since it can lower it at will if the requested length is against
  server policy.
\parameter{length}{The lifetime requested, measured in seconds.}

\result
\begin{description}
\item[0] If there is an error.
\item[$<>$ 0] otherwise.
\end{description}

\bigskip\begin{tabular}{lp{3in}}
VERR\_NOINIT & The \verb|vomsdata| structure was not properly
initialized.\\
\end{tabular}

\subsection{void VOMS\_Destroy(struct vomsdata *vd)}

This function destroys an allocated \verb|vomsdata| structure. It is
the ONLY supported way to do so.  It is also safe to pass a NULL
pointer to it.

\result
None.

\subsection{int VOMS\_Ordering(char *order, struct vomsdata *vd, int *error)}

This function is used to request a specific ordering of the attributes
present in an AC returned by the \verb|VOMS_Contact()| or by the
\verb|VOMS_ContactRaw()| functions. 

This function can be called several times, each time specifying a new
attribute.  The attributes in th AC created by the server will be in
the same order as the calls to this function, ignoring attributes
specified by this function that the server does not wish to grant.
Attributes not explicitly specified in this list will be inserted, in
an unspecified order, after all the others.

Never calling this function means that the corresponding list will be
empty, and as a consequence all the attributes will be in an
unspecified ordering.
\parameter{order}{The name of an attribute, in the
  $<$group$>$[:$<$role$>$:} format.

\begin{compatibility}
This is the only point where the FQAN format is not yet fully
supported.  Expect this to change in future revisions.
\end{compatibility}

\result
\begin{description}
\item[0] If there is an error.
\item[$<>$ 0] otherwise.
\end{description}

\errors
\bigskip\begin{tabular}{lp{3in}}
VERR\_NOINIT & The \verb|vomsdata| structure was not properly
initialized.\\
VERR\_PARAM  & The \verb|order| parameter is NULL.\\
VERR\_MEM    & There is not enough memory.\\
\end{tabular}


\subsection{int VOMS\_ResetOrder(struct vomsdata *cd, int *error)}

This function resets the attribute ordering set by the
\verb|VOMS_Ordering| function.

\result
\begin{description}
\item[0] If there is an error.
\item[$<>$ 0] otherwise.
\end{description}

\bigskip\begin{tabular}{lp{3in}}
VERR\_NOINIT & The \verb|vomsdata| structure was not properly
initialized.\\
\end{tabular}


\subsection{int VOMS\_Contact(char *hostname, int port, char
  *servsubject, char *command, struct vomsdata *vd, int *error)}

This function is used to contact a VOMS server to receive an AC
containing the calling user's authorization informations.  A
prerequisite to calling this function is the existance of a valid
proxy for the user himself.  This function does not create such a
proxy, which then must already exist.  Also, the parameters needed to
call this function should have been obtained by calling one of
\verb|FindByAlias()| or \verb|FindByVO()|.
\parameter{hostname}{This is the hostname of the machine hosting the
  server.}
\parameter{port}{This is the port number on which the server is
  listening.}
\parameter{servsubject}{This is the subject of the VOMS server'
  certificate.  This is needed for the mutual authentication.}
\parameter{command}{This is the command to be sent to the server.  For
  more info about it, consult the \verb|voms-proxy-init()| manual.}

\result
\begin{description}
\item[0] If there is an error.
\item[$<>$0] otherwise.  Furthermore, the data returned by the server
  has been parsed and added to the \verb|vomsdata| structure.
\end{description}

\errors
\bigskip\begin{tabular}{lp{3in}}
VERR\_NOINIT   & If the vomsdata structure was not properly
initialized.\\
VERR\_NOSOCKET & If it was impossible to contact the server.\\
VERR\_MEM      & If there was not enough memory.\\
VERR\_IDCHECK  & If a proxy certificate was not found or the data
returned by the server did not contain identifying information.\\
VERR\_FORMAT   & If there was an error in the format of the data
received.\\
VERR\_NODATA   & If no data was receied at all. (Usually as a
consequence of either a server error or not being recognized by the
server as a valid user.)\\
VERR\_ORDER    & If the attribute that the client requested, via the
\verb|VOMS_Ordering()| function, to be first in the list of attributes
received is not first in the attributes returned by the server. This
particular code means that the data has been correctly interpreted and
is available in the vomsdata structure if you want to use it.\\
VERR\_SERVERCODE & Some strange error occured in the server.\\
\end{tabular}


\subsection{int VOMS\_ContactRaw(char *hostname, int port, char *servsubject, char *command, void **data, int *datalen, int *version, struct vomsdata *vd, int *error)}

This function, like \verb|VOMS_Contact()| can be used to contact a
server and receive Authorization info from it.  The difference between
the two functions is that this version does not interpret the raw
data, but on the contrary returns it to the caller.  This function has
all the same prerequisites as \verb|VOMS_Contact()|.
\parameter{hostname}{This is the hostname of the machine hosting the
  server.}
\parameter{port}{This is the port number on which the server is
  listening.}
\parameter{servsubject}{This is the subject of the VOMS server'
  certificate.  This is needed for the mutual authentication.}
\parameter{command}{This is the command to be sent to the server.  For
  more info about it, consult the \verb|voms-proxy-init()| manual.}
\parameter{data}{A pointer to a pointer to an area of memory where the
  data returned from the server is stored.  It is the caller's
  responsibility to \verb|free()| this memory when it is no longer
  useful.}
\parameter{datalen}{The length of the data returned.}
\parameter{version}{The version of the AC returned.  Note that this is
  a \emph{minimum} version, it only guarantees that the data is \emph{at
  least} in that version of the format.}

\result
\begin{description}
\item[0] If there is an error.
\item[$<>$0] otherwise.  Furthermore, the data returned by the server
  has been parsed and added to the \verb|vomsdata| structure.
\end{description}

\errors
\bigskip\begin{tabular}{lp{3in}}
VERR\_NOINIT   & If the vomsdata structure was not properly
initialized.\\
VERR\_NOSOCKET & If it was impossible to contact the server.\\
VERR\_MEM      & If there was not enough memory.\\
VERR\_IDCHECK  & If a proxy certificate was not found or the data
returned by the server did not contain identifying information.\\
VERR\_FORMAT   & If there was an error in the format of the data
received.\\
VERR\_NODATA   & If no data was receied at all. (Usually as a
consequence of either a server error or not being recognized by the
server as a valid user.)\\
VERR\_ORDER    & If the attribute that the client requested, via the
\verb|VOMS_Ordering()| function, to be first in the list of attributes
received is not first in the attributes returned by the server. This
particular code means that the data has been correctly interpreted and
is available in the vomsdata structure if you want to use it.\\
VERR\_SERVERCODE & Some strange error occured in the server.\\
\end{tabular}


\subsection{int VOMS\_Retrieve(X509 *cert, STACK\_OF(X509) *chain, int how, struct vomsdata *vd, int *error)}

This function is used to extract from a proxy certificate the
VOMS-specific extension, to parse them and to insert the results into
the \verb|vomsdata| structure.
\parameter{cert}{This is the certificate that contains the VOMS
  information.  No checks are done on the validity of this certifiate,
  that is supposed to have already been verified by some other means.}
\parameter{chain}{This is the chain of certificates that signed the
  \verb|cert| certificate.  This pointer may be null, but see the next
  parameter.}
\parameter{how}{This parameter indicates how the search for the VOMS
  info will be performed.  If \verb|RECURSE_CHAIN| then the
  information is searched first into the \verb|cert| and then, if it
  was not found, in the walking the \verb|chain|, from the
  certificates to the CA.  If \verb|RECURSE_NONE| is specified, then
  the information is only searched in the \verb|cert|.

  In case the first value is specified, then the searches stop as soon
  as the info is found, ignoring further extension that may be found
  down the chain.}

\result
\begin{description}
\item[0] If there is an error.
\item[$<>$0] otherwise.  Furthermore, the data returned by the server
  has been parsed and added to the \verb|vomsdata| structure.
\end{description}

\errors
\bigskip\begin{tabular}{lp{3in}}
VERR\_NOINIT   & If the vomsdata structure was not properly
initialized.\\
VERR\_PARAM    & If there is something wrong with one of the parameters.\\
VERR\_MEM      & If there was not enough memory.\\
VERR\_IDCHECK  & If a proxy certificate was not found or the data
returned by the server did not contain identifying information.\\
VERR\_FORMAT   & If there was an error in the format of the data
received.\\
VERR\_NOEXT    & If the extension was not found.\\
\end{tabular}


\subsection{int VOMS\_Import(char *buffer, int buflen, struct vomsdata *vd, int *error)}

This function is used to add a string created with
\verb|VOMS_Export()| back into the vomsdata structure.
\parameter{buffer}{A pointer to the string.}
\parameter{buflen}{The length of the string.}

\result
\begin{description}
\item[0] If there is an error.
\item[$<>$0] otherwise.  Furthermore, the data returned by the server
  has been parsed and added to the \verb|vomsdata| structure.
\end{description}

\errors
\bigskip\begin{tabular}{lp{3in}}
VERR\_NOINIT   & If the vomsdata structure was not properly
initialized.\\
VERR\_FORMAT   & If there was an error in the format of the data
received.\\
VERR\_PARAM    & If there is something wrong with one of the parameters.\\
VERR\_MEM      & If there was not enough memory.\\
VERR\_IDCHECK  & If a proxy certificate was not found or the data
returned by the server did not contain identifying information.\\
VERR\_SERVER   & The VOMS server was unidentifiable.\\
VERR\_PARSE    & There has been some problem in parsing the AC or
blob.\\
VERR\_SIGN     & It was not possible to verify the signature.\\
VERR\_SERVER   & It was not possible to properly identify the Attribute
Issuer.\\
VERR\_TIME     & The check on the validity dates failed.\\
\end{tabular}



\subsection{int VOMS\_Export(char **buffer, int *buflen, struct
  vomsdata *vd, int *error)}

This function will take the current \verb|vomsdata| structure and
encode it in a string that can then be exported.
\parameter{buffer}{A pointer to an area of memory that will be
  allocated and filled by the function.  It is the caller's
  responsibility to \verb|free()| this memory.  It is possible that
  this pointer will be set to NULL, in case the \verb|vomsdata|
  structure is empty.}
\parameter{buflen}{The size of the data pointed by \verb|buffer|.}

\result
\begin{description}
\item[0] If there is an error.
\item[$<>$0] otherwise.  Furthermore, the data returned by the server
  has been parsed and added to the \verb|vomsdata| structure.
\end{description}

\errors
\bigskip\begin{tabular}{lp{3in}}
VERR\_PARAM    & If there is something wrong with one of the parameters.\\
VERR\_MEM      & If there was not enough memory.\\
\end{tabular}



\subsection{struct voms *VOMS\_DefaultData(struct vomsdata *vd, int
  *error)}

This function returns the default attributes from a vomsdata class.

\result
\begin{description}
\item[NULL] There has been an error or the \verb|vomsdata| structure
  was empty.
\item[$<>$NULL] There is some data.
\end{description}

\errors
\bigskip\begin{tabular}{lp{3in}}
VERR\_NOINIT & The \verb|vomsdata| structure was not properly
initialized.\\
VERR\_NONE   & The \verb|vomsdata| structure was empty.\\
\end{tabular}

\subsection{char *VOMS\_ErrorMessage(struct vomsdata *vd, int error, 
  char *buffer, int len)}

This function gives a textual description of the \emph{last}
encountered error.
\parameter{error}{The error returned by the previous function.}
\parameter{buffer}{A pointer to a buffer that will hold the error
  message.  If this is NULL, then it will be allocated by the function
  (and must be released by the caller).}
\parameter{len}{The length of the buffer pointed to by the previous
  parameter.}

\result
\begin{description}
\item[NULL] The buffer passed was not long enough, or there is not
  enough memory to allocate a buffer or the vomsdata structure was
  improperly initialized. 
\item[$<>$NULL] A pointer to a buffer containig the error message.  If
  \emph{buffer} was not null, then this is \emph{buffer}, else it is a
  newly allocated chunk of memory that should be free()ed by the
  caller.
\end{description}

\errors
\bigskip\begin{tabular}{lp{3in}}
VERR\_NOPARAM & The \verb|vomsdata| structure was not properly
initialized.\\
\end{tabular}

\subsection{int VOMS\_RetrieveEXT(X509\_EXTENSION *ext, struct vomsdata
  *vd, int *error)}

This function retrieves VOMS information from the given extension.
Due to the lack of a holder certificate, all checks regarding holder
information will be skipped.
\parameter{ext}{The extension to parse.}

\result
\begin{description}
\item[0] If there is an error.
\item[$<>$0] otherwise.  Furthermore, the data returned by the server
  has been parsed and added to the \verb|vomsdata| structure.
\end{description}

\errors
Check the description of the the VOMS\_Retrieve() function for a
description of the errors.


\subsection{int VOMS\_RetrieveFromCtx(gss\_ctx\_id\_t ctx, int how, struct vomsdata
  *vd, int *error)}

This function retrieves VOMS information from the given Globus
context.
\parameter{ctx}{The context from which to retrieve the certificate
  to parse.}
\parameter{how}{This parameter indicates how the search for the VOMS
  info will be performed.  If \verb|RECURSE_CHAIN| then the
  information is searched first into the \verb|cert| and then, if it
  was not found, in the walking the \verb|chain|, from the
  certificates to the CA.  If \verb|RECURSE_NONE| is specified, then
  the information is only searched in the \verb|cert|.

  In case the first value is specified, then the searches stop as soon
  as the info is found, ignoring further extension that may be found
  down the chain.}

\result
\begin{description}
\item[0] If there is an error.
\item[$<>$0] otherwise.  Furthermore, the data returned by the server
  has been parsed and added to the \verb|vomsdata| structure.
\end{description}

\errors
Check the description of the the VOMS\_Retrieve() function for a
description of the errors.

\subsection{int VOMS\_RetrieveFromCred(gss\_cred\_id\_t cred, int how, struct vomsdata
  *vd, int *error)}

This function retrieves VOMS information from the given Globus
credential.
\parameter{cred}{The credential from which to retrieve the certificate
  to parse.}
\parameter{how}{This parameter indicates how the search for the VOMS
  info will be performed.  If \verb|RECURSE_CHAIN| then the
  information is searched first into the \verb|cert| and then, if it
  was not found, in the walking the \verb|chain|, from the
  certificates to the CA.  If \verb|RECURSE_NONE| is specified, then
  the information is only searched in the \verb|cert|.

  In case the first value is specified, then the searches stop as soon
  as the info is found, ignoring further extension that may be found
  down the chain.}

\result
\begin{description}
\item[0] If there is an error.
\item[$<>$0] otherwise.  Furthermore, the data returned by the server
  has been parsed and added to the \verb|vomsdata| structure.
\end{description}

\errors
Check the description of the the VOMS\_Retrieve() function for a
description of the errors.

\subsection{int VOMS\_RetrieveFromProxy(int how, struct vomsdata *vd, int *error)}

This function retrieves VOMS information from an existing Globus proxy
certificate.
\parameter{how}{This parameter indicates how the search for the VOMS
  info will be performed.  If \verb|RECURSE_CHAIN| then the
  information is searched first into the \verb|cert| and then, if it
  was not found, in the walking the \verb|chain|, from the
  certificates to the CA.  If \verb|RECURSE_NONE| is specified, then
  the information is only searched in the \verb|cert|.

  In case the first value is specified, then the searches stop as soon
  as the info is found, ignoring further extension that may be found
  down the chain.}

\result
\begin{description}
\item[0] If there is an error.
\item[$<>$0] otherwise.  Furthermore, the data returned by the server
  has been parsed and added to the \verb|vomsdata| structure.
\end{description}

\errors
Check the description of the the VOMS\_Retrieve() function for a
description of the errors.


\end{document}
