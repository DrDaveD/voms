\documentclass[a4paper]{book}
\usepackage{color}
\usepackage{listings}
\newenvironment{compatibility}{\begin{quote}\color{red}Compatibility\begin{quote}}{\end{quote}\color{black}\end{quote}}
\newcommand{\also}[1]{\textbf{SEE ALSO}\newline\ \ \ \ \ #1}
\newcommand{\return}{\textbf{RETURNS}\newline}
\newcommand{\parameter}[2]{\newline\textbf{#1}\ \ #2}
\begin{document}
\lstset{language=C++}
\begin{titlepage}
\title{The VOMS C++ API\\ A Developer's Guide}
\author{Vincenzo Ciaschini}
\end{titlepage}
\maketitle
\tableofcontents
\newpage
\chapter{Introduction}
The VOMS API already come with their own documentation in doxygen
format.  However, that documentation is little more than a simple
enumeration of functions, with a very terse description.

The aim of this document is different.  Here the intention is not only
to describe the different functions that comprise the API, but also to
show how they are supposed to work together, what particular care the
user needs to take when calling them, what should be done to mantain
compatibility between the different versions, etc\ldots

Throughout this whole document, you will find sections marked thus:
\begin{compatibility}
Some information
\end{compatibility}
These section contain informations regarding both back and forward
compatibility between different versions of the API.

\begin{compatibility}
Finally, please note that everything not explicitly defined in this
argument should be considered a private detail and subject to change
without notice.
\end{compatibility}

\chapter{The API.}
There are three basic data structures: \verb|data|\/, \verb|voms|
and \verb|vomsdata|.

\section{The data structure}
The first one, \verb|data| contains the data regarding a single
attribute, giving its specification in terms of Groups, Roles and
Capabilities. It is defined as follows:

\begin{lstlisting}{}
struct data {
  std::string group; 
  std::string role;  
  std::string cap;   
};
\end{lstlisting}

All the values of these strings must be composed from regular
expression: \texttt{[a-ZA-Z0-9\_/]*}.

\subsection{group}
This field contains the name of a group which the user belongs into.
The format of entries in this group is reminiscent of the structure of
pathnames, and is the following:
\begin{quote}
\begin{emph}
/group/group/.../group
\end{emph}
\end{quote}
where the name of the first group is by convention the name of the
Virtual Organization (VO), while each other \emph{/group} component is
a subgroup of the group immediately preceding it on the left. The
character '/' is not acceptable as part of a group name.

This field MUST always be filled.

\subsection{role}
This field contains the name of the role which the user owns in the
group specified by \texttt{group}.  If the user does not own any
particular role in that group, than this field contains the value
``NULL''.

\subsection{cap}
This field details a capability that the user has as a member of the
group specified by \texttt{group} while owning the role specified by
\texttt{role}.  If there is no specific capability, than this value is
``NULL''. 

No specific format is associated to a capability.  They are basically
free-form strings, whose value should be agreed between the AA and the
Attribute verifier.

\section{The voms structure}
The second one, \texttt{voms} is used to group together all the
informations that can be gleaned from a single AC, and is defined as
follows:

\begin{lstlisting}{}
enum data_type { 
  TYPE_NODATA,  /*!< no data */
  TYPE_STD,     /*!< group, role, capability triplet */
  TYPE_CUSTOM   /*!< result of an S command */
};

struct voms {
  friend class vomsdata;
  int version;             
  int siglen;              
  std::string signature;   
  std::string user;        
  std::string userca;      
  std::string server;      
  std::string serverca;    
  std::string voname;      
  std::string uri;         
  std::string date1;       
  std::string date2;       
  data_type type;          
  std::vector<data> std;   
  std::string custom;      
  /* Data below this line only makes sense if version >= 1 */
  std::vector<std::string> fqan; 
  std::string serial;      
  /* Data below this line is private. */
private:
  AC *ac; 
  X509 *holder;
public:
  voms(const voms &);
  voms();
  voms &operator=(const voms &);
  ~voms();
};
\end{lstlisting}

The purpose of this structure is to present, in a readable format, the
data that has been included in a single Attribute Certificate
(AC).  While the various public fields may be freely modified to
simplify internal coding, such changes have no effect on the
underlying AC.  Let's examine the various fields in detail, starting
with the constructors.

\subsection{version}
This field specifies the version of this structure that is currently
being used.  A value of 0 indicates that it comes from an old format
extension, while a value of 1 indicates that this structure comes from
an AC.

\begin{compatibility}
Support for version 0 is going to be phased out of the code base in
roughly 6 months (late june - start of july).  When that happens,
version 0 structures will not be readable anymore. Until then, support
for it is being kept as a transition measure.

Update: With software version 1.6.0 and onwards, support for version 0
has been dropped. 
\end{compatibility}

Please do note that modifying the fields of a version 0 structure
associated with a \texttt{versiondata} object invalidates the result
of the \texttt{Export} method on that object.

\subsection{siglen}
The length of the data signature.

\subsection{user}
This field contains the subject of the holder's certificate in
slash-separated format.

\subsection{userca}
This field contains the subject of the CA that issued the holder's
certificate, in slash-separated format.

\subsection{server}
This field contains the subject of the certificate that the AA used to
issue the AC, in slash-separated format.

\subsection{serverca}
This field contains, in slash-separated format, the subject of the CA that
issued the certificate that the AA used to issue the AC.

\subsection{voname}
This field contains the name of the Virtual Organization (VO) to which
the rest of the data contained in this structure applies to.

\subsection{uri}
This is the URI at which the AA that issued this particular AC can be
contacted. Its format is:
\begin{quote}
\emph{fqdn}:\emph{port}
\end{quote}
where \emph{fqdn} is the Fully Qualified Domain Name of the server
which hosts the AA, and \emph{port} is the port at which the AA can
be contacted on that server.

\subsection{date1, date2}
These are the dates of start and end of validity of the rest of the
informations.  They are in a string representation readable to humans,
but they may be easily converted back to their original format, with a
little twist: dates coming from an AC are in GeneralizedTime format,
while dates coming from the old version data are in UtcTime format.

Here follows a code example doing that conversion:\bigskip\bigskip
\begin{lstlisting}{}
ASN1_TIME *
convtime(std::string data)
{
  ASN1_TIME *t= ASN1_TIME_new();

  t->data   = (unsigned char *)(data.data());
  t->length = data.size();
  switch(t->length) {
  case 10:
    t->type = V_ASN1_UTCTIME;
    break;
  case 15:
    t->type = V_ASN1_GENERALIZEDTIME;
    break;
  default:
    ASN1_TIME_free(t);
    return NULL;
  }
  return t;
}
\end{lstlisting}

\subsection{type}
This datum specifies the type of data that follows.  It can assume the
following values:
\begin{description}
\item [TYPE\_NODATA] There actually was no data returned.
\begin{compatibility}
This is actually only true for version 0 structures. The following
versions will simply not generate a \texttt{voms} structure in this
case.
\end{compatibility}
\item [TYPE\_CUSTOM] The data will contain the output of an ``S''
  command sent to the server.
\begin{compatibility}
Again, this type of datum will only be present in version 0
structures.  Due to lack of use, support for it has been disabled in
new versions of the server.
\end{compatibility}
\item [TYPE\_STD] The data will contain (group, role, capabilities)
  triples.
\end{description}

\subsection{std}
This vector contains all the attributes found in an AC, in the exact
same order as they were found, in the format specified by the
\texttt{data} structure.  It is only filled if the value of the
\texttt{type} field is \texttt{TYPE\_STD}.
\begin{compatibility}
This structure is filled in both version 1 and version 0 structures,
although this is scheduled to be left empty after the transition
period has passed.
\end{compatibility}

\subsection{custom}
This field contains the data returned by the ``S'' server command, and
it is only filled if the \texttt{type} value id \texttt{TYPE\_CUSTOM}.

\subsection{fqan}
This field contains the same data as the \texttt{std} field, but
specified in the Fully Qualified Attribute Name (FQAN) format.

\section{vomsdata}
The purpose of this object is to collect in a single place all
informations present in a VOMS extension. It is defined so.

\begin{lstlisting}{}
struct vomsdata {
  private:
  class Initializer {
  public:
    Initializer();
  private:
    Initializer(Initializer &);
  };

  private:
  static Initializer init;
  std::string ca_cert_dir;
  std::string voms_cert_dir;
  int duration;
  std::string ordering;
  std::vector<contactdata> servers;
  std::vector<std::string> targets;

  public:
  verror_type error; /*!< Error code */

  vomsdata(std::string voms_dir = "", 
	   std::string cert_dir = "");
  bool LoadSystemContacts(std::string dir = "");
  bool LoadUserContacts(std::string dir = "");
  std::vector<contactdata> FindByAlias(std::string alias);
  std::vector<contactdata> FindByVO(std::string vo);
  void Order(std::string att);
  void ResetOrder(void);
  void AddTarget(std::string target);
  std::vector<std::string> ListTargets(void);
  void ResetTargets(void);
  std::string ServerErrors(void);
  bool Retrieve(X509 *cert, STACK_OF(X509) *chain, 
		recurse_type how = RECURSE_CHAIN);
  bool Contact(std::string hostname, int port, 
               std::string servsubject, 
               std::string command);
  bool ContactRaw(std::string hostname, int port, 
		  std::string servsubject, 
		  std::string command,
		  std::string &raw, int &version);
  void SetVerificationType(verify_type how);
  void SetLifetime(int lifetime);
  bool Import(std::string buffer);
  bool Export(std::string &data);
  bool DefaultData(voms &);
  std::vector<voms> data;
  std::string workvo;
  std::string extra_data;

  std::string ErrorMessage(void);
  bool RetrieveFromCtx(gss_ctx_id_t context, recurse_type how);
  bool RetrieveFromCred(gss_cred_id_t credential, recurse_type how);
  bool Retrieve(X509_EXTENSION *ext);
  bool RetrieveFromProxy(recurse_type how);

private:
  /* not relevant: removed from this listing. */
};
\end{lstlisting}

Let us see the fields in detail.

\subsection{error}
This field contains the error code returned by one of the methods.
Please note that the value of this field is only significant if the
\emph{last} method called returns an error value.  Also, the value of
this field is subject to change without notice during method
executions, regardless of whether an error effectively occurred.

The possible values returned are the following:\bigskip\bigskip

\begin{lstlisting}{}
enum verror_type { 
  VERR_NONE,
  VERR_NOSOCKET,  
  VERR_NOIDENT,   
  VERR_COMM,      
  VERR_PARAM,     
  VERR_NOEXT,     
  VERR_NOINIT,    
  VERR_TIME,      
  VERR_IDCHECK,   
  VERR_EXTRAINFO, 
  VERR_FORMAT,    
  VERR_NODATA,    
  VERR_PARSE,     
  VERR_DIR,       
  VERR_SIGN,      
  VERR_SERVER,    
  VERR_MEM,       
  VERR_VERIFY,    
  VERR_TYPE,      
  VERR_ORDER,     
  VERR_SERVERCODE 
};
\end{lstlisting}

In general, a first idea of what each code means can be gleaned from
the code name, but in any case every method description will document
what errors its execution may generate and on which conditions.

\subsection{data}
This field contains a vector of \texttt{voms} structures, in the exact
same order as the corresponding ACs appeared in the proxy certificate,
and containing the informations present in that AC.

\section{Methods}
\subsection {voms}
\subsection{voms::voms()}
This is the standard default constructor.  Please note that a structure
created this way would not contain any real data.  The only use for this
constructor is to create a ``placeholder'' structure to which you
will copy data using the copy operator.

\subsection{voms::voms(const voms \&)}
This is the standard copy constructor.  Structures allocated via this
method will retain an exact copy of the data of their source.

\subsection{voms::operator=(const voms \&)}
This defines an assignment operator between two different
\texttt{voms} structures.

\section{vomsdata}
\subsection{vomsdata::vomsdata(std::string voms\_dir='''',
    std::string cert\_dir='''')}

This is the standard constructor that also doubles as the default
constructor. 
\parameter{voms\_dir}{This is the directory where the VOMS server'
certificates are kept. If this value is empty (the default), then the
value of \texttt{\$X509\_VOMS\_DIR} is considered, and if this is also
empty than its default is \texttt{/etc/grid-security/vomsdir}.}
\parameter{cert\_dir}{This is the directory where the CA certificates
are kept. If this value is empty (the default), then the value of
\texttt{\$X509\_CERT\_DIR} is considered, and if this is also empty
than its default is \texttt{/etc/grid-security/certificate}.}

\begin{compatibility}
This function is the only supported way to create and initialize a
\texttt{vomsdata} structure other than the copy constructor. It is
forbidden to ever take the \texttt{sizeof()} of this class.
\end{compatibility}

The default values are strongly suggested.  If you want to hardcode
specific ones, think very hard about the loss of configurability that
it would entail.

\subsection{bool vomsdata::LoadSystemContacts(std::string dir = ``'')}
This function loads the vomses files that are shared system-wide.
\parameter{dir}{This is the directory in which the various vomses files are
kept.  If left as blank, it defaults to \texttt{\$PREFIX/etc/vomses}.}

\return
The return value is true if all went well and false otherwise.  In the
latter case the \texttt{vomsdata::error} member becomes significant,
and it may assume the following values:

\bigskip\begin{tabular}{lp{3in}}
VERR\_DIR & The function tried to access something that either was not
a directory or a regular file, could not be read, or it had the wrong
permissions.  The correct permissions are 644 for files and 755 for
directories.\\
VERR\_FORMAT & The file was not in the expected format.
\end{tabular}

\subsection{bool vomsdata::LoadUserContacts(std::string dir = ``'')}
This function loads the vomses files that are user-specific.
\parameter{dir}{This is the directory in which the various vomses
files are kept.  If left as blank, it defaults to
\texttt{\$VOMS\_USERCONF}.  If this is also empty, then the last
default is \texttt{\$HOME/.edg/vomses}.}

\return
The return value is true if all went well and false otherwise.  In the
latter case the \verb|vomsdata::error| member becomes significant,
and it may assume the following values:

\bigskip\begin{tabular}{lp{3in}}
\color{black}VERR\_DIR & \color{black}The function tried to access
something that either was not a directory or a regular file, could not
be read, or it had the wrong permissions.  The correct permissions are
644 for files and 755 for directories.\\ 
VERR\_FORMAT & The file was not in the expected format.\\
\end{tabular}

\subsection{std::vector$<$contactdata$>$ vomsdata::FindByAlias(std::string alias)}

\begin{lstlisting}{}
struct contactdata {   /*!< You must never allocate directly this structure.
			    Its sizeof() is subject to change without notice.
			    The only supported way to obtain it is via the
			    FindBy* functions. */
  std::string nick;    /*!< The alias of the server */
  std::string host;    /*!< The hostname of the server */
  std::string contact; /*!< The subject of the server's certificate */
  std::string vo;      /*!< The VO served by this server */
  int    port;	       /*!< The port on which the server is listening */
};
\end{lstlisting}

This function looks in the vomses files loaded by
\texttt{vomsdata::LoadSystemContacts()} and
\texttt{vomsdata::LoadUserContacts()} for servers that have been
registered with a particular alias.
\parameter{alias}{The alias that will be searched for.  The search will be case sensitive.}

\return The return value is a vector containing the data (in
\texttt{contactdata} format) of all the servers known by the system
that go by the specified alias.  This function does not have an error
code, but the vector may be empty if no servers satisfying the query
are found or if there are no known servers altogether, typically
because the Load*Contacts() function have not been called.

\bigskip\subsection{void vomsdata::Order(std::string attribute)}

This function should be called before the various Contact*() ones, and
it is used to specify in which order the clients would like to
have the attributes returned by the server.

It can be called multiple times, each time specifying a new attribute,
creating in this way an ordered list of attributes.  Then, when the
server is contacted, it will exemine this list of attributes against
the one it would grant the client, and order the latter in the same
way, with the following provisions:
\begin{itemize}
\item All attributes not explicitly indicated in the order list will
  be placed in an unspecified order after all the specified ones.
\item An attribute present in the order list but not present among the
  attributes that the server is prepared to grant will be silently
  ignored.
\end{itemize}
\ \parameter{attribute}{The attribute that should be ordered}

\begin{compatibility}
For the moment, this is the only place where the FQAN format for
attribute names is not yet fully supported. The attribute field will
so have to be specified in the $<$group name$>$:$<$role name$>$
format.  This situation will be corrected sometime in the 1.2.x
series.
\end{compatibility}
\also{ResetOrder}

\subsection{void vomsdata::ResetOrder(void)}

This function clears the list of attributes that has been setup via
calls to the Order() function.
\also{Order}

\subsection{void vomsdata::AddTarget(std::string target)}

This function takes advantage of ACs capability to target themselves
to a specific set of hosts.  Through consecutive calls of this
function, the user can target the AC that the server will generate to
any set of hosts it likes.  Obviously, this funciton should be called
before the Contact*() ones.
\parameter{target}{The name of the host to which the AC will be targeted.  The name MUST be expressed in Fully Qualified Host Name format.}
\also{ListTargets, ResetTargets}

\subsection{std::vector$<$std::string$>$ vomsdata::listTargets(void)}

function returns a vector containing the list of hosts that will
constitute the targets that will be include in the AC.

\return A vector whose members are the FQHNs of the machines against
which the AC will be targeted.  This may be empty if the list has been
cleared or it has never been filled.

\also{AddTarget, ResetTargets}

\subsection{void vomsdata::ResetTargets(void)}

This function clears the list of targets for an AC.
\also{AddTarget, ListTargets}

\subsection{std::string vomsdata::ServerErrors()}

In case one of the other functions returned a \texttt{VERR\_SERVER}
message, meaning that some error has occurred on the server side of a
connection, calling this function MAY return a message from the server
itself detailing the error.

\return The error message itself

\subsection{void vomsdata::SetVerificationType(verify\_type how)}

This function sets the type of AC verification done by the Retrieve() and
Contact() functions.  The choices are detailed in the verify\_type
type.

\begin{lstlisting}{}
enum verify_type {
  VERIFY_FULL      = 0xffffffff,
  VERIFY_NONE      = 0x00000000,
  VERIFY_DATE      = 0x00000001,
  VERIFY_TARGET    = 0x00000002,
  VERIFY_KEY       = 0x00000004,
  VERIFY_SIGN      = 0x00000008,
  VERIFY_ORDER     = 0x00000010,
  VERIFY_ID        = 0x00000020
};
\end{lstlisting}

The meaning of these types is the following:
\begin{description}
\item[VERIFY\_DATE] This flag verifies that the current date is
  within the limits specified by the AC itself.
\item[VERIFY\_TARGET] This flag verifies that the AC is being
  evaluated in a machine that is included in the target extension of
  the AC itself.
\item[VERIFY\_KEY] This flag is for a future extension and is unused at the
  moment. 
\item[VERIFY\_SIGN] This flag verifies that the signature of the AC
  is correct.
\item[VERIFY\_ORDER] This flag verifies that the attributes present
  in the AC are in the exact order that was requested.  Please note
  that this can ONLY be done when examining an AC right after
  generation with the Contact() function.  This flag is meaningless
  in all other cases.
\item[VERIFY\_ID] This flag verifies that the holder information
  present in the AC is consistent with:
\begin{enumerate}
\item The enveloping user proxy in case the AC was contained in one.
\item The user's own certificate in case the AC was received without
  an enclosing proxy.
\end{enumerate}
\item[VERIFY\_FULL] This flag implies all other verifications.
\item[VERIFY\_NONE] This flag disables all verifications.
\end{description}

These flags can be combined by OR-ing them together.  However, if
VERIFY\_NONE is OR-ed to any other flag, it can be dismissed, while if
VERIFY\_FULL is OR-ed to any other flag, all other flags ca be
dismissed.

If this function is not explicitly called by the user, a VERIFY\_FULL
flag is considered to be in effect.


\subsection{void vomsdata::SetLifetime(int lifetime)}

This function should be called before the Contact*() ones.  Its aim is
to set the requested lifetime for the AC that the server would
create.  Please note that this is only a suggestion, and that the
server may well override it if the requested time is against its own
policy.
\parameter{lifetime}{The requested lifetime, in seconds.}

\subsection{bool vomsdata::Retrieve(X509 *cert, STACK\_OF(X509) *chain, recurse\_type how = RECURSE\_CHAIN)}\label{ret}

This function retrieves a VOMS AC from a VOMS-enabled proxy
certificate, executes the verifications requested by the
SetVerificationType() function and interprets the data.
\parameter{cert}{This is the X509 proxy certificate from which we want to
retrieve the informations.}
\parameter{chain}{This is the certificate chain associated to the proxy
certificate.  This parameter is only significant if the value of the
next parameter is \texttt{RECURSE\_CHAIN}.}
\parameter{how}{This parameters may have two values:
\begin{description}
\item[RECURSE\_NONE] meaning that the VOMS extension MUST be found in
  the certificate proper, or
\item[RECURSE\_CHAIN] meaning that if the VOMS extension are not found
  in the certificate proper, the certificate chain may be descended
  until either the extension is found or the chain ends.
\end{description}
The default value is \texttt{RECURSE\_CHAIN}.}

\texttt{RECURSE\_NONE} should only be used in special circumstances,
since it is guaranteed that in a normal Grid environment the process
of credential delegation will make the VOMS extension to be only
present in the certificate chain.


The result value is a boolean that is \texttt{true} if and only if
there have not been errors.  If the value is \texttt{false}, then you
should check the error code, which may have one of the following
values:


\bigskip\begin{tabular}{lp{3in}}
VERR\_PARAM   & There was something wrong with the parameters passes to
the function, or some of the required information (holder, etc...) is
empty.\\ 
VERR\_FORMAT  & If the format of the data is unknown (e.g. neither an
AC nor an old-style blob.\\
VERR\_NOIDENT & If it was impossible to discover the holder of the AC.\\ 
VERR\_NOINIT  & The vomsdata object hasn't been properly initialized.
Most likely the voms\_dir and ca\_dir parameters are empty.\\
VERR\_PARSE   & There has been some problem in parsing the AC or
blob.\\
VERR\_VERIFY  & It was not possible to verify the signature.\\
VERR\_SERVER  & It was not possible to properly identify the Attribute
Issuer.\\
VERR\_TIME    & The check on the validity dates failed.\\
VERR\_IDCHECK & The holder of the AC is not the same entity as the
holder of the enclosing certificate.\\
\end{tabular}

\also{SetVerificationType()}

\subsection{bool vomsdata::Contact(std::string hostname, int port, std::string servsubject, std::string command)}

This function is used to contact a specified server and use the
received AC to fill the vomsdata structure.
\parameter{hostname}{The fully qualified hostname of the machine on which the server runs.}
\parameter{port}{The port number on which the server is listening.}
\parameter{servsubject}{The subject of the server' certificate.}
\parameter{command}{The command to be sent to the server.}

These parameters may be obtained by using the FindByAlias() and
FindByVO() methods.

\return The return value is \texttt{true} if everything went well,
\texttt{false} otherwise.  In the latter case, the error field becomes
significant, and it may assume the following values.


\bigskip\begin{tabular}{lp{3in}}
VERR\_NOSOCKET & The client was unable to connect to the server.\\
VERR\_COMM    & Some communication errors (Usually related to
certificate problems)\\
VERR\_SERVERCODE & The server returned an error code.  More detailed
information may be obtaind by the ServeError() function.\\
VERR\_PARAM   & There was something wrong with the parameters passed to
the function, or some of the required information (holder, etc...) is
empty.\\ 
VERR\_FORMAT  & If the format of the data is unknown (e.g. neither an
AC nor an old-style blob.\\
VERR\_NOIDENT & If it was impossible to discover the holder of the
AC or the client was unable to find its own proxy certificate.\\ 
VERR\_NOINIT  & The vomsdata object hasn't been properly initialized.
Most likely the voms\_dir and ca\_dir parameters are empty.\\
VERR\_PARSE   & There has been some problem in parsing the AC or
blob.\\
VERR\_VERIFY  & It was not possible to verify the signature.\\
VERR\_SERVER  & It was not possible to properly identify the Attribute
Issuer.\\
VERR\_TIME    & The check on the validity dates failed.\\
VERR\_IDCHECK & The holder of the AC is not the same entity as the
holder of the enclosing certificate.\\
\end{tabular}

\subsection{bool vomsdata::ContactRaw(std::string hostname, int port,
  std::string servsubject, std::string command, std::string \&raw, int
  \&version)}

This function is used to contact a specified server and use the
received AC to fill the vomsdata structure.
\parameter{hostname}{The fully qualified hostname of the machine on which the server runs.}
\parameter{port}{The port number on which the server is listening.}
\parameter{servsubject}{The subject of the server' certificate.}
\parameter{command}{The command to be sent to the server.}
\parameter{raw}{This is an output parameter, and it will contain the data received by the server.}
\parameter{version}{This, too, is an output parameter, and it will contain the version number of the data included.}

The first four parameters may be obtained by using the FindByAlias() and
FindByVO() methods.

\return The return value is \texttt{true} if everything went well,
\texttt{false} otherwise.  In the latter case, the error field becomes
significant, and it may assume the following values.


\bigskip\begin{tabular}{lp{3in}}
VERR\_NOSOCKET & The client was unable to connect to the server.\\
VERR\_COMM    & Some communication error (Usually related to
certificate problems)\\
VERR\_SERVERCODE & The server returned an error code.  More detailed
information may be obtaind by the ServeError() function.\\
VERR\_PARAM   & There was something wrong with the parameters passed to
the function, or some of the required information (holder, etc...) is
empty.\\ 
VERR\_FORMAT  & If the format of the data is unknown (e.g. neither an
AC nor an old-style blob.\\
VERR\_NOIDENT & If the client was unable to find its own proxy certificate.\\ 
VERR\_NOINIT  & The vomsdata object hasn't been properly initialized.
Most likely the voms\_dir and ca\_dir parameters are empty.\\
\end{tabular}

\subsection{bool vomsdata::Export(std::string \&data)}

This function is used to create a string representation of all the
data that has been read from VOMS certificates so far.
\parameter{data}{This is an output parameter, and it will contain the data in encoded format.}

\return The return value is \texttt{true} if everything went well,
\texttt{false} otherwise.  In the latter case, the error field becomes
significant, and it may assume the following values.


\bigskip\begin{tabular}{lp{3in}}
VERR\_MEM    & There is not enough memory free.\\
VERR\_FORMAT & There is an inconsistency in the internal data.\\
VERR\_TYPE   & The same as above.  The difference is only for debugging
purposes.\\
\end{tabular}
\also{Import()}

\subsection{bool vomsdata::Import(std::string buffer)}

This function is used to add a string created by the Export() call
back into the vomsdata structure.  This function also runs
verification again.
\parameter{buffer}{The string to convert.}

\return The return value is \texttt{true} if everything went well,
\texttt{false} otherwise.  In the latter case, the error field becomes
significant, and it may assume the following values:


\bigskip\begin{tabular}{lp{3in}}
VERR\_PARAM   & There was something wrong with the parameters passes to
the function, or some of the required information (holder, etc...) is
empty.\\ 
VERR\_FORMAT  & If the format of the data is unknown (e.g. neither an
AC nor an old-style blob.\\
VERR\_NOIDENT & If is was impossible to discover the holder of the
AC or there was not a user certificate ready.\\ 
VERR\_NOINIT  & The vomsdata object hasn't been properly initialized.
Most likely the voms\_dir and ca\_dir parameters are empty.\\
VERR\_PARSE   & There has been some problem in parsing the AC or
blob.\\
VERR\_VERIFY  & It was not possible to verify the signature.\\
VERR\_SERVER  & It was not possible to properly identify the Attribute
Issuer.\\
VERR\_TIME    & The check on the validity dates failed.\\
VERR\_IDCHECK & The holder of the AC is not the same entity as the
holder of the enclosing certificate.\\
\end{tabular}

\subsection{bool vomsdata::DefaultData(voms \&d)}

This function returns the default attributes from a vomsdata class.
\parameter{d}{This is the \texttt{voms} structure that will contain
the default attributes.}

\return The return value is \texttt{true} if everything went well,
\texttt{false} otherwise.  In the latter case, the error field becomes
significant, and it may assume the following values:

\bigskip\begin{tabular}{lp{3in}}
VERR\_NOEXT & If there was no default attributes (most likely because
no attributes were read in.\\
\end{tabular}

\subsection{std::string vomsdata::ErrorMessage(void)}

This function returns a textual description for the error encountered
by the other functions.  This cannot fail.

\subsection{bool vomsdata::RetrieveFromCtx(gss\_ctx\_id\_t context,
  recurse\_type how)}

This function is capable of retrieving VOMS AC information from a GSS
context.
\parameter{context}{The context from which to obtain the certificate.}
\parameter{how}{What to do with the certificate chain. See the
documentation of Retrieve (\ref{ret}) for possible values.}

Return and error values are the same as Retrieve.  Again, see
(\ref{ret}) for possible values.

\subsection{bool vomsdata::RetrieveFromCred(gss\_cred\_id\_t credential,
  recurse\_type how)}

This function is capable of retrieving VOMS AC information from a GSS
credential.
\parameter{credential}{The credential from which to obtain the certificate.}
\parameter{how}{What to do with the certificate chain. See the
documentation of Retrieve (\ref{ret}) for possible values.}

Return and error values are the same as Retrieve.  Again, see
(\ref{ret}) for possible values.

\subsection{bool vomsdata::Retrieve(X509\_EXTENSION *ext)}

Tihs function retrieves the VOMS AC extension from the passed
extension.  Please note that the unavailability of the holder
certificate means that checks related to the holder of the AC will not
be done.
\parameter{ext}{The extension to evaluate.}

Return and error values are the same as Retrieve.  Again, see
(\ref{ret}) for possible values.


\subsection{bool vomsdata::RetrieveFromProxy(recurse\_type how)}

This function is capable of retrieving VOMS AC information from an
existing proxy.
\parameter{how}{What to do with the certificate chain. See the
documentation of Retrieve (\ref{ret}) for possible values.}

Return and error values are the same as Retrieve.  Again, see
(\ref{ret}) for possible values.

\end{document}
